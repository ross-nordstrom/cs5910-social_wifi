\section{Introduction}
\label{section:introduction}
With the steady growth of wireless networks as the core component of our connectedness, the study
and application of WiFi authentication is pervasive. Any improvements to the efficacy or ease of use
of WiFi authentication can have an enormouse impact.

In this paper I present the existing methods of authentication for Wireless networks (specifically
WLANs). In general, they fall into two categories targeting (1) Residential and Small Office
networks and (2) Enterprise networks. While each suffers for different reasons, I propose both would
benefit from integration with OAuth providers such as Google, Facebook, Github, Twitter, or a
private OAuth server. Offloading authentication to these providers contributes to both small and
enterprise networks in different ways, while improving the user experience in both cases.

\subsection{Residential and Small Networks}
The first type of network suffers heavily from lack of understanding of wireless networks and
``the human factor,'' which introduces vulnerabilities not inherent to the authentication protocol.
In these networks, there are likely to be relatively few users, each with relatively many devices.
Most of these devices will already be authenticated to social OAuth providers, which would allow
for wireless network authentication software to prompt the user to simply reuse that authentication
for access to the WLAN.

These small networks are the primary target of this proposal, as they stand to gain the most through
the ease of use introduced by socially-authenticated WLANs.

\subsection{Enterprise Networks}
The second type of network would benefit from integration with OAuth providers because it would
offload maintenance of yet another server(s). Additionally, more and more enterprises are using
OAuth for internal authentication, so this would allow them to reuse their main authentication
mechanism \cite{todo:OAuthEnterprise}.