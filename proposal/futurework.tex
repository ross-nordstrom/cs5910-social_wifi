\section{Future Work}
\label{section:futurework}
Future work on my idea will mainly take more time in order to complete a proof of concept
demonstrating OAuth-based access to a wifi network. Since it's already been done with CoovaAp, the
major focus should be on ease of use, supporting several providers, and marketing the idea rather
than ability to do the authentication.

\subsection{Ease of Use}
Just as the original goal of my research was to improve security of WiFi networks by making it easy
for non-tech-savvy users to maintain secure networks, future work should respect this primary goal.
Ease of use of the system falls into two categories: (1) ease of setting up and maintaining the
networks, and (2) ease of authenticating to the network with a minimum of work

With the prevalence social authentication on users' devices, it should be feasible to reuse existing
authenticated sessions with identity providers (Facebook, Google, Twitter, etc...) so that users can
authenticate to the wireless network with minimal user input. It should be easy to streamline the
ease of use since Wireless Network authentication uses a browser page, which should automatically
reuse the device's authentication session, since that's core to the design of OAuth.

A more problematic space will be authenticating devices to the network. Luckily, Google has some
good documentation on how to handle device authentication with their implementation of OAuth 2
\cite{google:OAuthDevices}. However, research will have to be done on how to make it easy to
replicate device authentication with other services, like Facebook and Twitter.

\subsection{Support Major Providers}
Another challenge to deal with in future work is ensuring the system can integrate with most
providers users might want to work with. On paper, OAuth2 claims to standardize authentication
workflows regardless of provider; however, in practice each provider's integration looks a little
different. Future work will need to identify the most important authentication providers and ensure
support for each of those. One additional requirement should include making the system easily
extensible so enterprises could integrate with their own proprietary OAuth system.

Currently, the top candidates to target would be Facebook, Google+, Twitter, and Github. Twitter
and Github have the least-obvious architectures to enforce authorization of ``who can access my
wireless network,'' unlike Facebook and Google+ which support user-managed ``groups'' of friends.

\subsection{Marketing}
Finally, no system is of any use if it's unknown to the audience that might use it. Most of the
marketing work to ensure a system like this gains popularity will be from partnership with one or
more of the major router manufacturers like Linksys, D-Link, and Cisco. Additionally, the system
will be best-served by being made available as an Open Source project so that developers can
easily collaborate to extend the OAuth providers it integrates with, and ensure the system is
secure. As we know from the historical implementations of security systems, the best way to ensure
their success is by making them public so that as many people as possible can try to find holes in
the system.