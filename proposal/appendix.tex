\section{Appendix}
\label{section:appendix}

\subsection{CoovaAP}
The CoovaAP setup was fairly straight-forward, if flaky. As mentioned in the proposal, I used a
Linksys WRT54GL \cite{product:WRT54GL} with the corresponding CoovaAP binary, available on their
public website \cite{product:CoovaAPBin}. After the firmware was installed (by using the router's
existing Firmware Update page), I had access to CoovaAP. I found that the Admin page allowing me to
configure the AP was very flaky in Chrome (version TODO) and Firefox (version TODO), but consistent
in Internet Explorer (version 11.0.9600).

Having learned that the Facebook integration is old and no longer works
\cite{article:CoovaFacebookNoMore}, I decided to continue with my original plan of implementing my
own captive portal server.

\subsubsection{Captive Portal Setup}
I managed to find a good tutorial \cite{article:CoovaHotSpotSetup} on setting up a custom captive
portal page and followed it for the following main steps:
\begin{enumerate}
\item{Clear RAM:}  To ensure a fresh configuration, SSH into the router with \texttt{ssh
root@192.168.1.1} (password ``root''), then run \texttt{ mtd erase nvram \&\& reboot }.
\item{Setup Hotspot:}  Back in the Admin page (browse to http://192.168.1.1 in IE), setup a
\texttt{HotSpot Type = Internal Hotspot} with \texttt{Registration Mode = ToS Acceptance} and Save.
\item{Fix Broken Portal:}  In the SSH session, open (with ViM for example)
\texttt{/et c/chilli/www/tos.chi}, then change \textbf{line 37}:
  \begin{enumerate}
  \item{from:} \texttt{ if [ "\$tos" = "1" ]; then }
  \item{to:} \texttt{ if [ "\$HS\_REG\_MODE" = "tos" ]; then}
  \end{enumerate}
\item{Confirm it worked:}  Now try to access the internet via the router (SSID: \texttt{Coova}), you
should see the Terms of Service page with Accept and Decline buttons at the bottom.
\end{enumerate}