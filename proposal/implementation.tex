\section{Implementation}
\label{section:implementation}
Given the scope of this project, I aimed to achieve a minimum viable prototype, leaving improvements
to get it actually adopted for Future Work\ref{section:futurework}.

\subsection{Proof of Concept Goals}
Here I break down the work I did to create an initial proof of concept. Note that all captive portal
communication would eventually be over SSL/TLS to ensure the credentials eventually passed to the
user cannot be read by eavesdroppers.

\subsubsection{Captive Portal}
The first thing I need to do is implement a captive portal on an OpenWRT router, where I can
redirect all users of the open SSID to an authentication page.

\subsubsection{Router-Facebook Integration}
The next thing I need to do is integrate the router with Facebook and authenticate it on behalf of
the owner in order to get a list of their friends. Additionally I need to securely store the OAuth
token allowing the router to authenticate for the owner to Facebook.

\subsubsection{User-Facebook Authentication}
The next step is to have users authenticate with Facebook so the captive portal server can learn
their identity and see if they are one of the owner's friends. If they are found to be in that list,
the server will present the user with the credentials to the secure SSID.

\subsubsection{Programmatic SSID Re-keying}
The final step in the proof of concept is to show that I can program the router to periodically
change the access credentials for the encrypted SSID.

\subsection{Proof of Concept Implementation}
It turns out there's an open source tool for instrumenting an OpenWRT router with a captive portal
and RADIUS. The product is CoovaAP \cite{product:CoovaAP}, and it can be deployed as a simple
firmware upgrade to OpenWRT-enabled routers. Specifically I worked with a Linksys WRT54GL
\cite{product:WRT54GL}.

An interesting and extremely relevant feature of CoovaAP is support for ``Facebook HotSpot Captive
Portal.'' Unfortunately, it has not been maintained and has not worked in several years
\cite{article:CoovaFacebookDead}. In lieu of this, I decided to try out integrating with Google+
OAuth as a proof of concepts. Luckily, CoovaAp support programming a HotSpot server, so I tried
instrumenting this. I was able to setup a Google Project with which to authenticate, but could not
get the Hotspot server to integrate with it in the time I had. Details on the code I wrote and
setup I did can be found in the appendix \ref{section:appendix}. Many of the challenges I faced
had to do with the documentation for using CoovaAp. In many cases, the documentation was outdated,
in other cases there simply wasn't the level of detail I could have used to improve my setup time.
