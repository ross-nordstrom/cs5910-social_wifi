\section{Implementation}
\label{section:implementation}
Given the scope of this project, I aimed to achieve a minimum viable prototype, leaving improvements
to get it actually adopted for Future Work\ref{section:futurework}.

\subsection{Proof of Concept}
Here I break down the work I did to create an initial proof of concept. Note that all captive portal
communication would eventually be over SSL/TLS to ensure the credentials eventually passed to the
user cannot be read by eavesdroppers.

\subsubsection{Captive Portal}
The first thing I need to do is implement a captive portal on an OpenWRT router, where I can
redirect all users of the open SSID to an authentication page.


\subsubsection{Router-Facebook Integration}
The next thing I need to do is integrate the router with Facebook and authenticate it on behalf of
the owner in order to get a list of their friends. Additionally I need to securely store the OAuth
token allowing the router to authenticate for the owner to Facebook.

\subsubsection{User-Facebook Authentication}
The next step is to have users authenticate with Facebook so the captive portal server can learn
their identity and see if they are one of the owner's friends. If they are found to be in that list,
the server will present the user with the credentials to the secure SSID.

\subsubsection{Programmatic SSID Re-keying}
The final step in the proof of concept is to show that I can program the router to periodically
change the access credentials for the encrypted SSID.
