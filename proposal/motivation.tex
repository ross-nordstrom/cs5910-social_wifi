\section{Motivation}
\label{section:motivation}
Existing WiFi authentication mechanisms, while technically secure, tend to create a poor user
experience and become vulnerable because of insecure behaviors of the users. Additionally,
most methods of authentication make it difficult for an administrator to revoke access to the
network, or even see who and what has access to begin with.

\subsection{Authentication Types}
There are several authentication techniques supported for communication over WiFi. They include
the following types of authentication, roughly in ascending order of security and grouped by security:
\cite{WifiAuthenticationTypes} \cite{wiki:WirelessSecurity}

\subsubsection{Insecure Types}
\begin{description}
\item{SSID Hiding:} Wireless networks can be ``hidden'' by not broadcasting the SSID; however,
it's extremely easy to detect hidden SSIDs. This is a ``security through obscurity'' technique,
which does a poor job of providing protection.
\item{Open WEP Authentication:} Devices must have a Wired Equivalent Privacy (WEP) key matching
the access point's WEP keys. These keys must be pre-shared, and any authenticated device can
immediately authenticate with the access point. Use of WEP has been deprecated since 2004 and has
been shown to be insecure.
\item{Shared-Key WEP Authentication:} Similarly, devices have a shared key, but now must
authenticate in through a four-step handashake. The device sends a request to the access point,
who responds with an unencrypted challenge text, which the device encrypts using the shared key
and returns to the AP. Finally, the access point decrypts the device's response and returns it.
This technique is actually less secure than Open WEP Authentication, but intruders can see both
the clear and encrypted text, and therefore derive the shared key. Once derived, they could use
the shared key for uninhibited access to the wireless network.
\item{MAC Address Authentication:} An access point can be configured with a white-list of allowed
device MAC addresses. Any device with a MAC address on that list can communicate over the network.
Authentication by MAC address is problematice for a couple reasons. First, it is onerous on the
system administrator to have to manage the list of allowed MAC addresses, especially at scale.
Secondly, it is actually pretty easy to spoof MAC addresses. An intruder simply needs to eavesdrop
on a successful authentication, and then assume that device's MAC address.
\end{description}

\subsubsection{Secure Types}
\begin{description}
\item{WPA Authentication:} WPA (Wi-Fi Protected Access) is a successor to WEP, addressing the
vulnerabilities of WEP that introduced insecurities. Specifically, it uses TKIP (Temporal Key
Integrity Protocol) to dynamically use a new key for each packet after initial authentication.
Additionally it replaces WEP's use of CRC (Cyclic Redundancy Check) with Michael for better packet
integrity verification. For all that, WPA suffers from two main vulnerabilities: (1) users can
choose to use insecure secrets, and (2) can be bypassed through Wi-Fi Protected Setup.

WPA comes in two primary types: WPA-Personal (or WPA-PSK, pre-shared key) and WPA-Enterprise. They
have different requirements and target audiences: \cite{wiki:WPA}
\begin{itemize}
\item{WPA-Personal:} Intended for residential and small office networks. All devices authenticate
with the same key. Also called ``WPA-PSK.''
\item{WPA-Enterprise:} Intended for enterprise networks, required a RADIUS authentication server.
It uses EAP for authentication. Also called ``WPA-802.1X mode'' or ``WPA.''
\end{itemize}

\item{WPA2:} Replaces WPA with the major contribution being CCMP, a stronger AES-based encryption.
CCMP stands for ``Counter Cipher Mode with block chaining message authentication code Protocol.''

\item{EAP Authentication:} The Extensible Authentication Protocol (EAP) defines a set of
interactions between the device and a RADIUS authentication server, with the access point
relaying. Once authenticated, the device gets a unique key for continued authenticated access. It
comes in a variety of types, with the most prominent being the PEAP (Protected EAP), and
\textbf{insecure} LEAP (Lightweight EAP).

\item{RF Shielding:} Rather than worry about intruders gaining unauthorized access to the network,
it can be simpler to coat the surrounding structure (e.g. room, office space, or building) with
a material, film, or paint that prevents wireless signals from entering or escaping.

\item{Token Identifiers:} Smart cards and USB/Software Tokens can be used for one of the most
secure authentication mechisms. The token provider (card, USB, or software) combines a stored
token with a user-entered PIN in a time-sensitive algorithm to frequently produce new encryption
codes. Authentication with the server requires the device to be synchronized with its clock.
\end{description}

\subsection{Shared Keys and Users}
Most of the authentication types targeting residential and small office rely on a shared key.
Regardless of the security of the authentication technique, they all suffer vulnerabilities from
the human element. From one angle, users could mismanage or misplace secrets, exposing the
credentials to intruders. Another problem is using secure passwords that are still easy to
remember is difficult to achieve. As hackers gain access to more and more leaked passwords, they
are able to study human patterns and behaviors in password use, rendering techincally secure
passwords insecure. However, if you use randomly generated passwords, the users will likely store
them somewhere, exposing the system to attack.

\subsection{RADIUS Authentication Servers}
The authentication types targeting enterprise use typically rely on a RADIUS authentication server.
RADIUS stands for ``Remote Authentication Dial In User Service,'' and provides AAA (Authentication,
Authorization, and Accounting) \cite{wiki:RADIUS}. Authentication with a RADIUS server is usually
with EAP or another authentication protocol. The user information can be stored locally in the
RADIUS server or integrated with an external store, such as SQL, Kerberos, LDAP, or Active
Directory.